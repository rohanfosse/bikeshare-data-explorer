%% ============================================================
%%  Mesure et Cartographie de la Mobilité Douce en France
%%  Standard : IEEE Transactions on Intelligent Transportation Systems
%%  (IEEEtran, journal mode)
%%  Compilation : pdflatex → bibtex → pdflatex × 2
%%  Encodage   : UTF-8
%% ============================================================
\documentclass[journal,10pt,twoside]{IEEEtran}

%% ── Encodage & Langue ───────────────────────────────────────
\usepackage[utf8]{inputenc}
\usepackage[T1]{fontenc}
\usepackage[french]{babel}
\usepackage{csquotes}

%% ── Mathématiques ───────────────────────────────────────────
\usepackage{amsmath,amssymb,bm}

%% ── Figures ─────────────────────────────────────────────────
\usepackage{graphicx}
\usepackage[export]{adjustbox}
\graphicspath{{figures/}}

%% ── Tableaux ─────────────────────────────────────────────────
\usepackage{booktabs}
\usepackage{multirow}
\usepackage{tabularx}
\usepackage{array}
\newcolumntype{C}[1]{>{\centering\arraybackslash}p{#1}}
\newcolumntype{L}[1]{>{\raggedright\arraybackslash}p{#1}}
\newcolumntype{R}[1]{>{\raggedleft\arraybackslash}p{#1}}

%% ── Couleurs ─────────────────────────────────────────────────
\usepackage[dvipsnames]{xcolor}
\definecolor{ieeblue}{RGB}{0,82,155}
\definecolor{corrgreen}{RGB}{39,174,96}
\definecolor{warnred}{RGB}{192,57,43}

%% ── Hyperliens ───────────────────────────────────────────────
\usepackage[hidelinks,unicode]{hyperref}
\hypersetup{colorlinks=true,linkcolor=ieeblue,citecolor=ieeblue,urlcolor=ieeblue}

%% ── Divers ───────────────────────────────────────────────────
\usepackage{microtype}
\usepackage{subcaption}
\usepackage{siunitx}
\sisetup{output-decimal-marker={,},group-separator={\,}}
\usepackage{enumitem}
\usepackage{rotating}

%% ── Métadonnées ─────────────────────────────────────────────
\title{Mesure et Cartographie de la Mobilité Douce en France~:\\
Un Indice Composite Socio-Spatial pour l'Analyse Comparative\\
des Systèmes de Vélos en Libre-Service}

\author{%
  \IEEEauthorblockN{Rohan~Fossé}
  \IEEEauthorblockA{%
    CESI\\
    Montpellier, France\\
    \texttt{rfosse@cesi.fr}}%
}

\markboth{IEEE TRANSACTIONS ON INTELLIGENT TRANSPORTATION SYSTEMS,~VOL.~XX,~NO.~XX,~2025}%
{Fossé~: Mesure et Cartographie de la Mobilité Douce en France}

% ─────────────────────────────────────────────────────────────
\begin{document}
\maketitle

%% ============================================================
\begin{abstract}
Les systèmes de vélos en libre-service (SVLS) représentent un levier
croissant des politiques de mobilité urbaine durable. Si leur déploiement
s'accélère en France, leur évaluation comparative souffre de deux faiblesses
récurrentes~: une hétérogénéité de qualité des données brutes et un recours
à des indicateurs mono-dimensionnels. Ce travail propose un cadre d'analyse
intégré. \emph{Premièrement}, un pipeline GBFS automatisé est couplé à un
audit qualité multicritère (cinq classes d'anomalies). \emph{Deuxièmement},
un \emph{Indice de Mobilité Douce} (IMD) à cinq composantes est calculé sur
\num{62}~agglomérations~; les pondérations sont calibrées empiriquement par
optimisation supervisée maximisant la corrélation de Spearman avec le
Baromètre FUB~2023 et l'EMP~2019 (évolution différentielle,
$\rho_{\text{FUB}} : 0{,}16 \rightarrow 0{,}47$, $\rho_{\text{EMP}} : 0{,}26
\rightarrow 0{,}48$), puis validées par Monte~Carlo ($N=10\,000$).
\emph{Troisièmement}, un \emph{Indice d'Équité Sociale} (IES) met en regard
la performance et le profil socio-économique local (INSEE 2019--2021).
Appliqué à \num{125}~systèmes GBFS couvrant \num{46359}~stations
géolocalisées, le modèle révèle une forte dispersion des performances
($\text{IMD} \in [0{,}25;\,0{,}79]$, médiane~$= 0{,}36$, $\sigma = 0{,}12$).
Le poids dominant $w_M^* = 0{,}578$ (multimodalité) induit une corrélation
modérée avec la taille démographique ($r_s = +0{,}41$, $p < 0{,}05$),
reflétant l'avantage infrastructurel des grandes métropoles.
La correction de l'anomalie~A3 (Pony) fait chuter Bordeaux du rang~2 au
rang~14 avec les poids normatifs, illustrant l'impact de la qualité des données.
Sur les \num{62}~agglomérations avec profil complet, le meilleur prédicteur
est la part des diplômés du supérieur ($r_s = +0{,}53$, $p < 0{,}001$)~;
le modèle Ridge ($R^2_{\text{train}} = 0{,}39$) laisse une fraction
substantielle inexpliquée, attribuable aux choix politiques locaux.
\end{abstract}

\begin{IEEEkeywords}
vélo libre-service, GBFS, qualité des données, indice composite, couverture
spatiale, équité sociale, mobilité douce, France, classification socio-spatiale
\end{IEEEkeywords}

%% ============================================================
\section{Introduction}
\label{sec:introduction}

\IEEEPARstart{L}{a} mobilité douce s'impose progressivement comme un axe
prioritaire des stratégies de transition écologique urbaine. En France, le
\emph{Plan Vélo 2023--2027} vise à tripler la part modale du vélo d'ici~2030,
tandis que la loi d'orientation des mobilités (LOM, 2019) contraint les
autorités organisatrices de mobilité (AOM) à intégrer le vélo dans leurs
offres multimodales~\cite{LOM2019}. Les systèmes de vélos en libre-service
(SVLS) constituent un maillon essentiel de cette chaîne multimodale
\cite{DeMaio2009}.

Pourtant, l'évaluation comparative de ces systèmes souffre de trois lacunes.
\emph{Premièrement}, les analyses existantes portent principalement sur des
villes isolées ou des métropoles étrangères, sans vision nationale cohérente
\cite{Fishman2016}. \emph{Deuxièmement}, les indicateurs mobilisés restent
mono-dimensionnels et ne capturent ni la couverture spatiale effective ni
l'accessibilité sociale~\cite{ElGeneidy2016}. \emph{Troisièmement}, la
qualité variable des flux GBFS --- standard pourtant adopté par la quasi-%
totalité des opérateurs français --- n'a jamais été systématiquement auditée
dans la littérature, ce qui rend tout classement comparatif potentiellement
trompeur.

Ce travail répond à quatre questions de recherche~:
\begin{description}[leftmargin=1.6em,labelsep=0.4em]
  \item[\textbf{Q1.}] Quelles anomalies de données affectent les flux GBFS
    français, et dans quelle mesure biaisent-elles les classements de
    performance~?
  \item[\textbf{Q2.}] Comment construire un indice composite de performance
    des SVLS à la fois robuste, multi-dimensionnel et interprétable~?
  \item[\textbf{Q3.}] Existe-t-il une corrélation systématique entre les
    performances des SVLS et le profil socio-économique des agglomérations
    françaises~?
  \item[\textbf{Q4.}] Quels facteurs socio-économiques prédisent le mieux
    l'IMD, et quelle est la part résiduelle attribuable aux choix politiques~?
\end{description}

Les contributions de ce travail sont au nombre de six~:
\begin{enumerate}[leftmargin=1.8em,itemsep=0pt]
  \item un \textbf{pipeline GBFS} automatisé avec audit qualité multicritère~;
  \item une \textbf{taxonomie des anomalies GBFS} françaises (5~classes)~;
  \item l'\textbf{Indice de Mobilité Douce (IMD)}, composite à 5~composantes~;
  \item une \textbf{calibration empirique des poids} par optimisation supervisée
    (FUB + EMP) et validation Monte~Carlo ($N = 10\,000$)~;
  \item l'\textbf{Indice d'Équité Sociale (IES)}, performance ajustée du
    contexte socio-économique~;
  \item une \textbf{cartographie nationale} des déserts de mobilité et des
    typologies de villes.
\end{enumerate}

%% ============================================================
\section{État de l'Art}
\label{sec:relatedwork}

\subsection{Systèmes de vélos en libre-service}

Les travaux fondateurs de DeMaio~\cite{DeMaio2009} et Shaheen~\emph{et
al.}~\cite{Shaheen2010} posent une taxonomie (dock-based, dockless,
électrique) et analysent les facteurs de succès. La recherche post-2015
s'articule autour de~: la modélisation de la demande
\cite{Singhvi2015,Lin2018}, le rééquilibrage des stations
\cite{Raviv2013,Forma2015}, et les déterminants socio-démographiques de
l'adoption~\cite{Buck2013,Wang2016}. La comparaison multi-villes reste rare~:
Gu~\emph{et al.}~\cite{Gu2019} analysent 20~villes chinoises par densité et
couverture~; Fishman~\cite{Fishman2016} offre une revue sans quantification
unifiée.

\subsection{Qualité des données GBFS}

La qualité des flux GBFS est un sujet émergent. MobilityData
\cite{MobilityData2024} maintient un validateur officiel (GBFS Validator)
vérifiant la conformité syntaxique des flux, mais sans audit sémantique
des valeurs. Hunter~\emph{et al.}~\cite{Hunter2021} identifient des
incohérences dans les données de disponibilité temps réel (Vélib', BIXI),
sans traiter la qualité des données statiques de stations. À notre
connaissance, aucune étude n'a réalisé d'audit systématique des capacités
déclarées dans les flux \texttt{station\_information.json} à l'échelle
nationale.

\subsection{Indices composites d'accessibilité}

El-Geneidy~\emph{et al.}~\cite{ElGeneidy2016} formalisent la notion de
\emph{cumulative opportunity measure}. Geurs et van~Wee~\cite{Geurs2004}
proposent un cadre à quatre composantes (infrastructure, localisation,
temporelle, individuelle). Birago~\emph{et al.}~\cite{Birago2017}
introduisent une mesure raster de la couverture spatiale des réseaux de
transport. Ces trois approches inspirent directement la construction de l'IMD.

\subsection{Équité sociale et SVLS}

Hodge~\cite{Hodge1995} établit le cadre conceptuel de la \emph{transit
equity}. Smith~\emph{et al.}~\cite{Smith2015} montrent que le réseau
Divvy (Chicago) surreprésente les quartiers aisés. En France, le rapport
Cerema~\cite{Cerema2022} souligne l'inégalité d'accès aux SVLS selon le
revenu. Ces résultats motivent la construction de l'IES.

%% ============================================================
\section{Données, Collecte et Contrôle Qualité}
\label{sec:data}

\subsection{Architecture GBFS et pipeline de collecte}

Le \emph{General Bikeshare Feed Specification} (GBFS v3.0)
\cite{MobilityData2024} définit une arborescence de flux JSON~: depuis
\texttt{gbfs.json} (annuaire) jusqu'à \texttt{station\_information.json}
(géométrie, capacité) et \texttt{station\_status.json} (disponibilité).
Deux agrégateurs référencent les systèmes français~: \emph{transport.data.%
gouv.fr} (catalogue officiel national) et \emph{MobilityData} (registre
mondial). Un pipeline Python (\texttt{fetch\_gbfs\_france.py}) déduplique
les entrées par normalisation des URL GBFS et parcourt la chaîne de flux
pour chaque système, en collectant les données statiques de station.

\subsection{Audit qualité des données GBFS}
\label{sec:audit}

L'audit révèle cinq classes d'anomalies critiques, synthétisées dans la
Table~\ref{tab:anomalies}.

\begin{table}[!t]
  \caption{Taxonomie des anomalies GBFS identifiées dans le corpus français}
  \label{tab:anomalies}
  \centering
  \small
  \begin{tabular}{@{}lL{3.2cm}c@{}}
    \toprule
    Classe & Description & Systèmes\\
    \midrule
    \textbf{A1} & Systèmes d'autopartage (Citiz) inclus dans GBFS, non cyclistes & 14 \\
    \textbf{A2} & Capacité fictive (placeholder): valeur constante non nulle
                  sur toutes les stations (ex. \texttt{pony\_Nice}: $c=100$) & 1 \\
    \textbf{A3} & Capacité calculée sur stations non nulles uniquement
                  (biais de la moyenne; ex. \texttt{pony\_bordeaux}: $\bar{c}_{\text{profil}}=15$
                  vs $\bar{c}_{\text{réel}}=0{,}03$) & 5 \\
    \textbf{A4} & Coordonnées GPS aberrantes ou permutées,
                  générant des bounding-boxes continentales & 2 \\
    \textbf{A5} & Systèmes hors France métropolitaine (DOM-TOM)
                  ou à périmètre régional ($A_{\text{bbox}} > \SI{50000}{km^2}$) & 6 \\
    \midrule
    Total & & 21 \\
    \bottomrule
  \end{tabular}
\end{table}

\subsubsection{Anomalie A3 --- impact sur les classements}

L'anomalie~A3 est la plus impactante. Plusieurs opérateurs Pony publient
dans leur flux GBFS une capacité nulle pour la quasi-totalité de leurs stations,
à l'exception de quelques ancres physiques ponctuelles. Le calcul du profil
système dans le script de collecte utilisait la \emph{moyenne conditionnelle}
(moyenne des valeurs non nulles), produisant des moyennes gonflées~:

\begin{equation}
  \bar{c}_{\text{profil}} = \frac{\sum_{i : c_i > 0} c_i}{\#\{i : c_i > 0\}}
  \quad \neq \quad
  \bar{c}_{\text{réel}} = \frac{\sum_{i=1}^{N} c_i}{N}
  \label{eq:cap_bias}
\end{equation}

Pour \texttt{pony\_bordeaux} : $\bar{c}_{\text{profil}} = 90/6 = 15$
(compatible avec le seuil dock-based $\geq 8$) vs
$\bar{c}_{\text{réel}} = 90/2996 \approx 0{,}03$ (free-floating). Cette
erreur classait à tort \num{2996}~stations Pony comme dock-based dans
l'agrégation bordelaise, portant le compte de stations dock de Bordeaux de
$225$ (velo-tbm uniquement) à $3\,221$, et son rang IMD de~14 à~2.
Après correction, Bordeaux retourne à sa juste position.

\subsubsection{Procédure de correction}

Les corrections appliquées sont~:
\begin{enumerate}[leftmargin=1.8em,itemsep=0pt]
  \item Exclusion des \num{14}~systèmes Citiz (autopartage voiture) ;
  \item Reclassement de \texttt{pony\_Nice} en free-floating
    ($c=100$ fictif $\rightarrow$ exclu des agrégations capacité) ;
  \item Recalcul de $\bar{c}_{\text{réel}}$ pour tous les systèmes
    (\ref{eq:cap_bias}) et reclassification en dock/semi/FF~;
  \item Suppression des stations hors France métropolitaine
    ($\varphi < 41°$ ou $\varphi > 52°$, $\lambda < -6°$ ou $\lambda > 10°$)~;
  \item Suppression des outliers GPS intra-système ($> 3\sigma$ du centroïde)~;
  \item Seuil minimal de $N_{\min} = 20$ stations dock par agglomération
    pour figurer dans le classement (exclut les systèmes trop petits pour
    des métriques géographiques stables).
\end{enumerate}

\subsection{Données socio-économiques}

Les indicateurs socio-économiques sont issus de l'INSEE~\emph{Filosofi}
2019--2021 et du \emph{Recensement de la Population} (RP~2020), pour
les sept variables décrites dans la Table~\ref{tab:socio_vars}.

\begin{table}[!t]
  \caption{Variables socio-économiques retenues (INSEE 2019-2021)}
  \label{tab:socio_vars}
  \centering
  \small
  \begin{tabular}{@{}clc@{}}
    \toprule
    Var. & Définition & Source \\
    \midrule
    $r$ & Revenu médian par unité de consommation (\euro/an) & Filosofi 2021 \\
    $u$ & Taux de chômage (\%) & RP 2020 \\
    $c$ & Part des cadres (\%) & RP 2020 \\
    $d$ & Part diplômés Bac+3 et plus (\%) & RP 2020 \\
    $v$ & Part des ménages sans voiture (\%) & RP 2020 \\
    $p$ & Taux de pauvreté au seuil de 60\,\% (\%) & Filosofi 2021 \\
    $b$ & Part vélo dom.-travail (\%) & RP 2020 \\
    \bottomrule
  \end{tabular}
\end{table}

\subsection{Statistiques descriptives du jeu de données final}

La Table~\ref{tab:dataset} synthétise les caractéristiques du corpus après
application de la procédure de qualité. La Fig.~\ref{fig:carte_nationale}
présente la distribution géographique des systèmes retenus.

\begin{table}[!t]
  \caption{Statistiques descriptives du jeu de données (après audit qualité)}
  \label{tab:dataset}
  \centering
  \small
  \begin{tabular}{@{}lS[table-format=6.0]@{}}
    \toprule
    Indicateur & {Valeur} \\
    \midrule
    Systèmes GBFS bruts collectés         & 125 \\
    Systèmes exclus (audit qualité)        &  21 \\
    Systèmes retenus dans l'analyse        & 104 \\
    \quad dont dock-based ($\bar{c} \geq 8$) &  62 \\
    \quad dont semi-dock ($2 \leq \bar{c} < 8$) & 11 \\
    \quad dont free-floating ($\bar{c} < 2$) &  31 \\
    Stations géolocalisées (France métr.) & 46359 \\
    Agglomérations classées (IMD, $N_{\text{dock+semi}} \geq 3$) & 62 \\
    Agglomérations avec profil socio-éco  &  62 \\
    Indicateurs socio-économiques          &   7 \\
    \bottomrule
  \end{tabular}
\end{table}

\begin{figure}[!t]
  \centering
  \includegraphics[width=\columnwidth]{fig01_carte_nationale}
  \caption{Distribution nationale des systèmes GBFS retenus après audit.
    La taille des marqueurs est proportionnelle au nombre de stations~;
    la couleur encode le type (dock-based, semi-dock, free-floating).}
  \label{fig:carte_nationale}
\end{figure}

%% ============================================================
\section{Méthodologie}
\label{sec:methodology}

\subsection{Indice de Mobilité Douce (IMD)}

L'IMD est un indice composite à cinq composantes, normalisées dans $[0,1]$
par transformation min-max avant agrégation pondérée~:

\begin{equation}
  \text{IMD} = \sum_{k \in \{S,E,D,P,M\}} w_k \, C_k,
  \quad \sum_k w_k = 1
  \label{eq:imd}
\end{equation}

\noindent avec $\mathbf{w}^* = (0{,}184;\,0{,}069;\,0{,}051;\,0{,}118;\,0{,}578)$,
calibrés empiriquement (voir Section~\ref{sec:calibration}).
La Table~\ref{tab:composantes} détaille les cinq composantes.

\begin{table}[!t]
  \caption{Composantes de l'IMD}
  \label{tab:composantes}
  \centering
  \small
  \begin{tabular}{@{}cL{3.6cm}c@{}}
    \toprule
    $C_k$ & Définition & Poids $w_k$ \\
    \midrule
    $S$ & Couverture spatiale (raster, $r{=}300$\,m, $g{=}150$\,m) & \textbf{0{,}184} \\
    $E$ & Équité~: $1 - \mathrm{Gini}(\mathbf{c})$ & 0{,}069 \\
    $D$ & Densité~: $\log(1 + n_{\text{dock}}/\text{km}^2)$ normalisé & 0{,}051 \\
    $P$ & Per capita~: $\log(1 + n_{\text{dock}}/\text{1000\,hab.})$ norm. & 0{,}118 \\
    $M$ & Multimodalité~: nb de types (dock, semi, FF) & \textbf{0{,}578} \\
    \bottomrule
  \end{tabular}
\end{table}

\subsubsection{Couverture spatiale $S$}

Pour chaque station $i$ projetée en coordonnées métriques locales $(x_i, y_i)$,
un disque de rayon $r = \SI{300}{\metre}$ est discrétisé sur une grille de
résolution $g = \SI{150}{\metre}$. La couverture est~:

\begin{equation}
  S = \frac{\left|\bigcup_{i=1}^{N} D_i\right|}{A_{\text{hull}} \,/\, g^2}
  \label{eq:coverage}
\end{equation}

\noindent où $A_{\text{hull}}$ est l'aire de l'enveloppe convexe des stations,
calculée après suppression des outliers GPS ($> 3\sigma$). Les cellules
couvertes sont dédupliquées par encodage entier~:
$\mathrm{key}_{ij} = I_x^{(j)} \times 10^5 + I_y^{(j)}$ (complexité
$\mathcal{O}(N \cdot |\mathcal{D}|)$).

\subsubsection{Équité $E$ et coefficient de Gini}

\begin{equation}
  \text{Gini}(\mathbf{c}) = \frac{\sum_{i,j} |c_i - c_j|}{2N \sum_i c_i},
  \quad E = 1 - \text{Gini}(\mathbf{c})
  \label{eq:gini}
\end{equation}

\subsection{Calibration empirique des poids IMD}
\label{sec:calibration}

Les poids de l'équation~(\ref{eq:imd}) ont été calibrés empiriquement via une
\emph{optimisation supervisée} (NB~25). La fonction objectif maximise la
moyenne des corrélations de rang de Spearman avec deux références de terrain~:

\begin{equation}
  \mathbf{w}^* = \underset{\mathbf{w}}{\arg\max}\;
    \frac{1}{2}\Bigl[
      \rho\!\bigl(\textstyle\sum_k w_k C_k,\,y_{\text{FUB}}\bigr)
      + \rho\!\bigl(\textstyle\sum_k w_k C_k,\,y_{\text{EMP}}\bigr)
    \Bigr]
  \label{eq:opt_weights}
\end{equation}

\noindent sous les contraintes $\sum_k w_k = 1$, $w_k \geq 0{,}05$, résolue
par \emph{évolution différentielle} (\texttt{scipy.optimize.differential\_evolution},
500~générations, population $15 \times 5$). Les références sont~: le
\textbf{Baromètre FUB~2023} (score citoyen de praticabilité vélo, $n = 34$
villes) et la \textbf{EMP~2019} (part modale vélo nationale, $n = 46$~villes).
La méthode CRITIC~\cite{Diakoulaki1995} (pondération par variance corrigée des
corrélations inter-composantes) a également été testée à titre de comparaison.
Les résultats sont synthétisés dans la Table~\ref{tab:poids_comparaison}.

\begin{table}[!t]
  \caption{Comparaison des méthodes de pondération~: corrélations de Spearman
    avec les références externes.}
  \label{tab:poids_comparaison}
  \centering
  \small
  \begin{tabular}{@{}lC{1.1cm}C{1.1cm}C{0.9cm}@{}}
    \toprule
    Méthode & $\rho_{\text{FUB}}$ & $\rho_{\text{EMP}}$ & $w_M$ \\
    \midrule
    Poids normatifs & 0{,}162 & 0{,}260 & 0{,}10 \\
    CRITIC~\cite{Diakoulaki1995} & 0{,}208 & 0{,}277 & 0{,}21 \\
    \textbf{Opt. combinée (retenu)} & \textbf{0{,}473} & \textbf{0{,}475} & \textbf{0{,}578} \\
    \bottomrule
  \end{tabular}
\end{table}

\textbf{Robustesse - Monte Carlo.} La stabilité des classements est vérifiée
par $N = 10\,000$ tirages de Dirichlet uniformes sur le simplexe (filtrés
$w_k \geq 0{,}05$). Les villes multimodales (Bordeaux, Marseille, Nantes,
Lyon) apparaissent dans le Top~10 dans plus de 83\,\% des tirages.
Une analyse OAT confirme que $C_M$ est la composante pivot (amplitude~IMD
$\approx 0{,}12$), tandis que $C_E$ et $C_D$ ont un impact marginal.

\subsection{Indice d'Équité Sociale (IES)}
\label{sec:ies}

L'IES mesure la sur- ou sous-performance d'une agglomération relativement
à son niveau socio-économique attendu. Un modèle de référence linéaire est
estimé sur le revenu médian $r$~:

\begin{equation}
  \widehat{\text{IMD}}(r) = \alpha + \beta r
  \label{eq:imd_ref}
\end{equation}

\begin{equation}
  \text{IES} = \frac{\text{IMD}_{\text{obs}}}{\widehat{\text{IMD}}(r)}
  \label{eq:ies}
\end{equation}

Un IES~$> 1$ indique une agglomération qui \emph{sur-performe} par rapport à
sa richesse médiane~; IES~$< 1$ révèle un sous-investissement.

\subsection{Régression Ridge avec validation croisée LOO}

Pour identifier les déterminants socio-économiques de l'IMD, une régression
Ridge est estimée~:

\begin{equation}
  \hat{\bm{\beta}} = \underset{\bm{\beta}}{\arg\min}
    \left\| \mathbf{y} - \mathbf{X}\bm{\beta} \right\|_2^2
    + \lambda \|\bm{\beta}\|_2^2
  \label{eq:ridge}
\end{equation}

\noindent où $\mathbf{X} \in \mathbb{R}^{n \times 7}$ est la matrice des
variables socio-économiques standardisées (StandardScaler, $n{=}31$), et
$\lambda^*$ est sélectionné par minimisation du RMSE LOO-CV.

\subsection{Analyse en quadrants et clustering $k$-means}

L'espace bidimensionnel revenu $\times$ IMD est partitionné en quatre
quadrants par les médianes. Un clustering $k$-means est appliqué sur le
profil joint $\mathbf{z}_i = [\text{IMD}_i, r_i, u_i, d_i, v_i, p_i]$,
avec $k^*$ sélectionné par score de silhouette~\cite{Rousseeuw1987}.

%% ============================================================
\section{Résultats}
\label{sec:results}

\subsection{Impact de l'audit qualité sur les classements}
\label{sec:impact_audit}

La correction de l'anomalie A3 est la plus impactante. La
Table~\ref{tab:bordeaux_correction} compare la situation de Bordeaux avant
et après correction.

\begin{table}[!t]
  \caption{Impact de la correction A3 sur Bordeaux}
  \label{tab:bordeaux_correction}
  \centering
  \small
  \begin{tabular}{@{}lccc@{}}
    \toprule
    & Avant correction & Après correction & $\Delta$ \\
    \midrule
    Stations dock comptées & 3\,221 & 225 & $-93\,\%$ \\
    Densité dock (st/km²) & 7{,}12 & 1{,}89 & $-73\,\%$ \\
    Dock/1\,000\,hab. & 12{,}20 & 0{,}85 & $-93\,\%$ \\
    IMD & 0{,}690 & 0{,}364 & $-47\,\%$ \\
    Rang national & 2 & 14 & $-12$ \\
    \bottomrule
  \end{tabular}
\end{table}

Les \num{2996}~stations de \texttt{pony\_bordeaux} (mean\_capacity $= 0{,}03$
après correction) étaient comptabilisées à tort dans les stations dock-based
de la ville. Cet exemple illustre que tout classement comparatif SVLS fondé
sur des données GBFS brutes, sans procédure d'audit, peut produire des
conclusions fallacieuses.

\subsection{Classement IMD national}

La Table~\ref{tab:top10bot5} présente les dix premières et trois dernières
agglomérations ($n = 62$). La distribution révèle une forte dispersion
($\sigma = 0{,}12$, plage $[0{,}25;\,0{,}79]$, médiane~$= 0{,}36$).

\begin{table*}[!t]
  \caption{Classement IMD corrigé~: 10 premières et 5 dernières agglomérations
    ($N \geq 20$ stations dock). Les composantes normalisées $C_k \in [0,1]$.}
  \label{tab:top10bot5}
  \centering
  \small
  \begin{tabular}{@{}clccccccc@{}}
    \toprule
    Rang & Agglomération
         & IMD
         & $C_1\,(S)$ & $C_2\,(E)$ & $C_3\,(D)$ & $C_4\,(P)$ & $C_5\,(M)$
         & $N_{\text{dock}}$ \\
    \midrule
    1  & Bordeaux                & 0{,}787 & 0{,}397 & 0{,}783 & 0{,}652 & 0{,}417 & 1{,}000 &   225 \\
    2  & Marseille               & 0{,}755 & 0{,}378 & 0{,}840 & 0{,}619 & 0{,}153 & 1{,}000 &   221 \\
    3  & Nantes                  & 0{,}662 & 0{,}786 & 0{,}809 & 1{,}000 & 0{,}220 & 0{,}667 &   124 \\
    4  & Lyon                    & 0{,}660 & 0{,}022 & 0{,}812 & 0{,}006 & 0{,}189 & 1{,}000 &   452 \\
    5  & Rennes                  & 0{,}621 & 0{,}631 & 0{,}857 & 0{,}836 & 0{,}155 & 0{,}667 &    57 \\
    6  & Lille                   & 0{,}607 & 0{,}244 & 0{,}743 & 0{,}228 & 0{,}279 & 1{,}000 &   268 \\
    7  & La Roche-sur-Yon        & 0{,}591 & 0{,}746 & 0{,}836 & 0{,}869 & 0{,}330 & 0{,}667 &    57 \\
    8  & Paris                   & 0{,}578 & 0{,}461 & 0{,}794 & 0{,}846 & 0{,}089 & 1{,}000 & 1\,507 \\
    9  & Belfort                 & 0{,}562 & 0{,}606 & 0{,}882 & 0{,}710 & 0{,}333 & 0{,}667 &    32 \\
    10 & Toulouse                & 0{,}562 & 0{,}273 & 0{,}769 & 0{,}310 & 0{,}176 & 1{,}000 &   434 \\
    \midrule
    \multicolumn{9}{c}{$\cdots$} \\
    \midrule
    60 & Saint-Nazaire           & 0{,}273 & 0{,}026 & 0{,}989 & 0{,}030 & 0{,}045 & 0{,}333 &     5 \\
    61 & Châtellerault           & 0{,}262 & 0{,}010 & 0{,}979 & 0{,}012 &    ---  & 0{,}333 &     0 \\
    62 & Montreuil-Bellay        & 0{,}246 & 0{,}007 & 0{,}721 & 0{,}047 &    ---  & 0{,}333 &     0 \\
    \bottomrule
  \end{tabular}
\end{table*}

\textbf{Bordeaux}~(\#1, IMD$=0{,}787$) est l'unique agglomération à
conjuguer les trois types de systèmes (dock + semi-dock + free-floating,
$C_5 = 1{,}00$) avec une équité élevée ($C_2 = 0{,}783$) et une densité
significative ($C_3 = 0{,}652$). \textbf{Marseille}~(\#2, IMD$=0{,}755$)
présente le score d'équité le plus élevé du Top~10 ($C_2 = 0{,}840$) et
bénéficie également d'une couverture multimodale complète.
\textbf{Nantes}~(\#3, IMD$=0{,}662$) atteint la densité maximale du corpus
($C_3 = 1{,}00$) et une couverture spatiale remarquable ($C_1 = 0{,}786$).
\textbf{Lyon}~(\#4) illustre un paradoxe~: couverture spatiale quasi nulle
($C_1 = 0{,}022$) mais IMD élevé grâce à la multimodalité complète et
à l'équité maximale de son réseau Vélo'v ($C_2 = 0{,}812$).

En bas de classement, les agglomérations partagent une multimodalité réduite
($C_5 = 0{,}333$, un seul type de système). \textbf{Saint-Nazaire}~(\#60)
dispose d'un réseau de seulement 5~stations dock à très faible couverture~;
\textbf{Châtellerault} et \textbf{Montreuil-Bellay} (\#61--\#62) ne
possèdent pas de stations dock physiques (uniquement semi-dock).

Avec les poids optimisés, la corrélation IMD/population est modérée et
significative ($r_s = +0{,}41$, $p < 0{,}05$, Spearman, $n = 59$),
reflétant l'avantage multimodal structurel des grandes métropoles ($w_M^*
= 0{,}578$). La Fig.~\ref{fig:classement_imd} illustre la décomposition
par composante et la relation IMD/population.

\begin{figure}[!t]
  \centering
  \includegraphics[width=\columnwidth]{fig04_classement_imd}
  \caption{(a)~Classement IMD corrigé (Top~20), décomposé par composante
    ($\mathbf{w}^* = (0{,}184;\,0{,}069;\,0{,}051;\,0{,}118;\,0{,}578)$,
    poids calibrés, NB~25). (b)~IMD en fonction de la population (échelle
    logarithmique). La corrélation modérée ($r_s = +0{,}41$, $p < 0{,}05$)
    reflète l'avantage multimodal des grandes métropoles.}
  \label{fig:classement_imd}
\end{figure}

\subsection{Analyse socio-économique}

La Fig.~\ref{fig:correlations} présente les corrélations de Spearman entre
l'IMD et les six variables socio-économiques sur les \num{62}~agglomérations.
Les corrélations les plus significatives sont~:
part des diplômés du supérieur ($r_s = +0{,}53$, $p < 0{,}001$, \textbf{***}),
ménages sans voiture ($r_s = +0{,}44$, $p < 0{,}001$, ***) et part vélo
domicile-travail ($r_s = +0{,}43$, $p < 0{,}001$, ***).
Le taux de chômage exerce un effet négatif significatif ($r_s = -0{,}27$,
$p < 0{,}05$, *). Le revenu médian reste corrélé positivement mais modérément
($r_s = +0{,}30$, $p < 0{,}05$). Ces résultats confirment que le capital
humain et les comportements modaux préexistants covarient davantage avec
la qualité des SVLS que la richesse brute.

\begin{figure}[!t]
  \centering
  \includegraphics[width=\columnwidth]{fig05_correlations}
  \caption{Corrélations de Spearman IMD $\times$ 7~variables socio-économiques
    ($n=31$). Les ellipses encodent la direction et l'intensité.}
  \label{fig:correlations}
\end{figure}

\subsection{Analyse en quadrants socio-mobilité}

La Table~\ref{tab:quadrants} présente la répartition des agglomérations
dans les quatre quadrants. La Fig.~\ref{fig:quadrants} visualise leur
distribution.

\begin{table}[!t]
  \caption{Répartition par quadrant (revenu médian $\times$ IMD, $n = 62$)}
  \label{tab:quadrants}
  \centering
  \small
  \begin{tabular}{@{}lcc@{}}
    \toprule
    Quadrant & $N$ & \% \\
    \midrule
    Villes Favorisées --- Bien Équipées & 20 & 32{,}3 \\
    Désert de Mobilité Sociale\dag      & 18 & 29{,}0 \\
    Sous-Investissement                 & 13 & 21{,}0 \\
    Mobilité Inclusive\ddag             & 11 & 17{,}7 \\
    \midrule
    Total & 62 & 100{,}0 \\
    \bottomrule
    \multicolumn{3}{l}{\footnotesize \dag Revenu faible + IMD faible.} \\
    \multicolumn{3}{l}{\footnotesize \ddag IMD élevé malgré revenu faible.} \\
  \end{tabular}
\end{table}

\begin{figure}[!t]
  \centering
  \includegraphics[width=\columnwidth]{fig06_quadrants}
  \caption{Analyse en 4~quadrants~: revenu médian (axe~$x$) $\times$
    IMD (axe~$y$). Les médianes sont matérialisées par des lignes en
    pointillé. Les déserts de mobilité sociale sont encadrés.}
  \label{fig:quadrants}
\end{figure}

\subsection{Indice d'Équité Sociale}

La Table~\ref{tab:ies} présente les cinq plus grands sur-performeurs et
sous-performeurs selon l'IES (médiane~$= 0{,}885$, $\sigma = 0{,}28$,
$n = 62$). \textbf{Marseille} (IES~$= 2{,}04$) se distingue comme le
premier sur-performeur malgré un revenu médian bas ($r = \SI{16500}{\euro/an}$)
et un taux de pauvreté élevé ($30\,\%$)~: une illustration d'une politique
de déploiement volontariste dans un contexte socio-économique contraint.
\textbf{Bordeaux}~(IES~$= 1{,}82$) et \textbf{Lille}~(IES~$= 1{,}61$)
complètent le trio des meilleurs sur-performeurs.

À l'opposé, \textbf{Montreuil-Bellay} (IES~$= 0{,}61$) et
\textbf{Annecy}~(IES~$= 0{,}67$) sous-performent relativement à leur
niveau de revenu. Le cas d'Annecy est notable~: agglomération aisée
($r \approx \SI{24000}{\euro/an}$) avec un usage vélo effectif élevé,
mais un réseau GBFS incomplet à la date de collecte.

\begin{table}[!t]
  \caption{IES~: 5 sur-performeurs et 5 sous-performeurs ($n = 31$)}
  \label{tab:ies}
  \centering
  \small
  \begin{tabular}{@{}lS[table-format=1.3]S[table-format=1.3]S[table-format=5.0]@{}}
    \toprule
    Agglomération & {IMD} & {IES} & {$r$ (\euro/an)} \\
    \midrule
    \multicolumn{4}{@{}l}{\emph{Sur-performeurs}} \\
    Marseille             & 0.755 & 2.039 & 16500 \\
    Bordeaux              & 0.787 & 1.819 & 20500 \\
    Lille                 & 0.607 & 1.605 & 17000 \\
    Nantes                & 0.662 & 1.452 & 22000 \\
    Belfort               & 0.562 & 1.427 & 18000 \\
    \midrule
    \multicolumn{4}{@{}l}{\emph{Sous-performeurs}} \\
    Châtellerault         & 0.262 & 0.680 & 20000 \\
    Tarbes                & 0.279 & 0.709 & 19500 \\
    Saint-Nazaire         & 0.273 & 0.666 & 20500 \\
    Annecy                & 0.309 & 0.666 & 24000 \\
    Montreuil-Bellay      & 0.246 & 0.613 & 21000 \\
    \bottomrule
  \end{tabular}
\end{table}

\subsection{Déserts de mobilité sociale}

Les \textbf{18~agglomérations} (29\,\%) classées dans le quadrant
\emph{Désert de Mobilité Sociale} présentent simultanément un revenu inférieur
à la médiane ($r < r_{\text{med}}$) ET un IMD inférieur à la médiane
($\text{IMD} < 0{,}36$). Ces territoires concentrent des publics captifs de
la mobilité sans alternative SVLS performante~--- un enjeu majeur de
politique publique. La Fig.~\ref{fig:deserts} visualise leur position.

\begin{figure}[!t]
  \centering
  \includegraphics[width=\columnwidth]{fig07_deserts}
  \caption{Déserts de mobilité sociale~: fragilité sociale (axe~$x$)
    $\times$ IMD (axe~$y$). Les marqueurs encadrés cumulent les deux critères.
    La taille est proportionnelle au taux de pauvreté.}
  \label{fig:deserts}
\end{figure}

\subsection{Facteurs prédictifs de l'IMD}

La régression Ridge ($\alpha = 0{,}1$, $R^2_{\text{train}} = 0{,}39$,
$n = 62$) identifie~:

\begin{itemize}[itemsep=0pt]
  \item $\hat{\beta}_{\text{cadres}} = +0{,}105$~: la part des cadres est
    le prédicteur positif le plus fort (proxy de marchés du travail tertiaires
    favorables au vélo)~;
  \item $\hat{\beta}_{\text{revenu}} = -0{,}064$~: l'effet revenu est négatif
    après contrôle des autres variables (revenu capturant partiellement
    l'étalement urbain)~;
  \item $\hat{\beta}_{\text{chômage}} = -0{,}051$~: le taux de chômage exerce
    un effet modérateur négatif.
\end{itemize}

\noindent Le $R^2_{\text{LOO-CV}}$ est non défini (modèle sans généralisation
stable sur $n = 62$ observations), ce qui signifie qu'une fraction
substantielle de la performance SVLS est déterminée par des facteurs non
capturés~--- gouvernance locale, qualité du maillage, culture cyclable.
Le $R^2_{\text{train}} = 0{,}39$ indique néanmoins que la socio-économie
prédit environ un tiers de la variance (Fig.~\ref{fig:ridge}).

\begin{figure}[!t]
  \centering
  \includegraphics[width=\columnwidth]{fig08_ridge}
  \caption{Régression Ridge LOO-CV ($n=31$)~: coefficients normalisés
    (barres) et intervalles de confiance par bootstrap
    (\num{1000}~réplications). La ligne en pointillé~: $\beta=0$.}
  \label{fig:ridge}
\end{figure}

\subsection{Clustering socio-mobilité}

Le score de silhouette est maximisé pour $k^* = 2$ clusters
(Fig.~\ref{fig:clustering}). La Table~\ref{tab:clusters} caractérise les
deux groupes. Le cluster~C1 (\num{13}~villes, IMD~$= 0{,}51$, revenu
médian~$\approx \SI{21850}{\euro}$) rassemble les métropoles multimodales
à fort capital humain. Le cluster~C0 (\num{49}~villes) regroupe la majorité
avec des profils socio-économiques plus fragiles et une multimodalité
plus limitée (IMD~$= 0{,}38$).

\begin{table}[!t]
  \caption{Caractérisation des clusters ($k^* = 2$, $n=31$)}
  \label{tab:clusters}
  \centering
  \small
  \begin{tabular}{@{}lccc@{}}
    \toprule
    Cluster & $N$ & IMD moy. & $r$ moy. (\euro) \\
    \midrule
    C0 --- Villes Fragiles          & 49 & 0{,}38 & 18\,341 & --- \\
    C1 --- Métropoles Multimodales  & 13 & 0{,}51 & 21\,846 & --- \\
    \bottomrule
  \end{tabular}
\end{table}

\begin{figure}[!t]
  \centering
  \includegraphics[width=\columnwidth]{fig09_clustering}
  \caption{Clustering $k$-means ($k^* = 2$)~: projection ACP~2D avec
    ellipses de confiance à \SI{95}{\percent} et profils radar normalisés
    des deux clusters.}
  \label{fig:clustering}
\end{figure}

%% ============================================================
\section{Discussion}
\label{sec:discussion}

\subsection{Apport principal~: l'audit qualité comme pré-requis}

Le résultat le plus saillant n'est pas un résultat de classement, mais
une mise en garde méthodologique~: tout travail comparatif fondé sur des
données GBFS doit inclure une procédure d'audit préalable. L'anomalie~A3
(\texttt{pony\_bordeaux}) illustre qu'une simple erreur de calcul de
moyenne conditionnelle suffit à faire passer une agglomération du rang~14
au rang~2. Ce type de biais est systémique pour tous les opérateurs de type
\emph{floating anchor}~--- dont la station GBFS ne représente pas un
vrai rack physique mais un point de dépôt autorisé~--- dès lors que
la quasi-totalité des capacités enregistrées sont nulles.

\subsection{Performance et taille urbaine}

Avec les poids calibrés ($w_M^* = 0{,}578$), la corrélation IMD/population
est modérée et significative ($r_s = +0{,}41$, $p < 0{,}05$), reflétant
l'avantage infrastructurel des grandes métropoles qui disposent d'une
offre multimodale plus complète (dock + semi-dock + FF). Ce résultat
contraste avec l'absence de corrélation observée sous poids normatifs, et
met en lumière un artefact de la pondération~: \emph{la corrélation
IMD/population est partiellement construite par le choix des poids}.
Nantes (\num{323000}~hab., \#3) surclasse Paris (\num{10,7}~millions,
\#8) grâce à sa multimodalité et à sa densité dock, illustrant que la
qualité du déploiement prime sur le volume brut de stations.

\subsection{Équité sociale et sur-performance}

L'IES révèle que le revenu médian est un prédicteur imparfait de la
performance SVLS. Le quadrant \emph{Mobilité Inclusive} (11~villes)
rassemble des agglomérations à revenus inférieurs à la médiane mais à IMD
élevé~--- illustration d'une politique de déploiement volontariste.
\textbf{Marseille}~(IES~$= 2{,}04$) est le cas le plus frappant~: revenu bas,
taux de pauvreté de $30\,\%$, mais réseau SVLS parmi les deux meilleurs de
France grâce à une couverture multimodale complète. À l'inverse, le quadrant
\emph{Sous-Investissement} (13~villes) inclut des agglomérations relativement
aisées qui n'ont pas converti leurs ressources en réseau SVLS performant.

\subsection{Cas Strasbourg~: biais de couverture GBFS}

Le cas Strasbourg mérite une attention particulière. La ville affiche le
taux de déplacements domicile-travail à vélo le plus élevé de l'échantillon
($b = 18\,\%$), mais son IMD ($0{,}10$) et son IES ($0{,}28$) sont les plus
bas. Cette contradiction apparente s'explique par une couverture GBFS
incomplète~: seule une fraction du réseau Vélhop est publiée via GBFS à la
date de collecte. Ce cas illustre une limitation fondamentale de toute
approche GBFS~: \emph{l'absence d'un système dans les flux GBFS ne signifie
pas l'absence du système sur le terrain}. Les comparaisons inter-villes
doivent intégrer cette asymétrie de publication dans leur interprétation.

\subsection{Limites}

\textbf{Snapshot temporel.} Les données GBFS reflètent l'état du réseau au
moment de la collecte. Les variations saisonnières et les évolutions récentes
ne sont pas capturées.
\textbf{Enveloppe convexe.} Le dénominateur de~(\ref{eq:coverage}) surestime
l'aire de service pour les réseaux non convexes (villes côtières, vallées).
Une isochrone routière ($\leq 15$~min à pied) serait plus représentative.
\textbf{Pondérations IMD.} Les poids sont calibrés empiriquement
(Section~\ref{sec:calibration})~; la validation Monte~Carlo ($N = 10\,000$
tirages) confirme leur robustesse. Toutefois, les références FUB et EMP
couvrent respectivement \num{34} et \num{46}~villes sur \num{62}, introduisant
un biais de sélection potentiel vers les grandes agglomérations.
\textbf{Couverture GBFS.} Certains réseaux (Vélhop Strasbourg, certains
systèmes municipaux anciens) ne publient pas de flux GBFS complets,
sous-estimant leur performance.
\textbf{Millésime socio-économique.} Les données INSEE 2019--2021 peuvent
être décalées par rapport à des systèmes déployés après 2022.

%% ============================================================
\section{Conclusion}
\label{sec:conclusion}

Ce travail propose un cadre d'analyse intégré~--- pipeline GBFS avec audit
qualité, IMD à poids calibrés (\ref{eq:imd}, \ref{eq:opt_weights}) et
IES (\ref{eq:ies})~--- pour la mesure et la cartographie comparative des
SVLS en France. Son application à \num{125}~systèmes sur \num{62}~agglomérations
livre cinq résultats principaux~:

\begin{enumerate}[leftmargin=1.8em,itemsep=2pt]
  \item \textbf{L'audit qualité est indispensable}~: l'anomalie~A3 (Pony) fait
    varier le rang de Bordeaux de~2 à~14 sous poids normatifs, soit une
    erreur de $-93\,\%$ sur le compte de stations dock~;
  \item \textbf{La calibration empirique des poids} multiplie par~3 la
    corrélation avec les pratiques cyclables réelles
    ($\rho_{\text{FUB}} : 0{,}16 \rightarrow 0{,}47$)~; $w_M^* = 0{,}578$
    identifie la multimodalité comme levier dominant, validé par Monte~Carlo~;
  \item \textbf{La corrélation IMD/population} est modérée ($r_s = +0{,}41$,
    $p < 0{,}05$), artefact partiel de la pondération multimodale~; Nantes
    (\#3, 323\,k~hab.) surclasse Paris (\#8, 10,7\,M~hab.)~;
  \item \textbf{Dix-huit déserts de mobilité sociale} (29\,\%) cumulent
    fragilité économique et sous-équipement~; ces territoires appellent une
    intervention prioritaire des politiques publiques~;
  \item \textbf{La socio-économie explique $\approx 39\,\%$} de la variance
    de l'IMD (Ridge, $R^2_{\text{train}} = 0{,}39$)~; les diplômés du
    supérieur et l'usage vélo préexistant sont les meilleurs prédicteurs.
\end{enumerate}

Quatre extensions sont envisagées~: (1)~Voronoï national pour la
cartographie des zones blanches~; (2)~modèle de diffusion spatiale par
graphe de proximité~; (3)~analyse temporelle des IMD 2020--2025 via
archives Transitland~; (4)~intégration GTFS pour mesure enrichie de~$C_M$.

%% ============================================================
\appendix
\section{Disponibilité du code et des données}
\label{app:repro}

L'ensemble du code (pipeline GBFS, scripts de correction, notebooks Jupyter
numérotés 19--23), des données collectées et des figures est disponible dans
le dépôt du projet \emph{BikeShare-Graph-Forecasting}. Les données GBFS sont
issues d'API publiques (Licence Ouverte Etalab v2.0~; Open Database
Licence~1.0). Les données INSEE sont publiques (Licence Etalab v2.0).

Le script de correction (\texttt{scripts/correct\_imd.py}) est documenté et
reproductible~: son exécution produit le fichier
\texttt{imd\_classement.csv} décrit dans la Section~\ref{sec:data}.

%% ============================================================
\bibliographystyle{IEEEtran}
\bibliography{references}

\end{document}
